% Options for packages loaded elsewhere
\PassOptionsToPackage{unicode}{hyperref}
\PassOptionsToPackage{hyphens}{url}
%
\documentclass[
]{article}
\usepackage{amsmath,amssymb}
\usepackage{lmodern}
\usepackage{iftex}
\ifPDFTeX
  \usepackage[T1]{fontenc}
  \usepackage[utf8]{inputenc}
  \usepackage{textcomp} % provide euro and other symbols
\else % if luatex or xetex
  \usepackage{unicode-math}
  \defaultfontfeatures{Scale=MatchLowercase}
  \defaultfontfeatures[\rmfamily]{Ligatures=TeX,Scale=1}
\fi
% Use upquote if available, for straight quotes in verbatim environments
\IfFileExists{upquote.sty}{\usepackage{upquote}}{}
\IfFileExists{microtype.sty}{% use microtype if available
  \usepackage[]{microtype}
  \UseMicrotypeSet[protrusion]{basicmath} % disable protrusion for tt fonts
}{}
\makeatletter
\@ifundefined{KOMAClassName}{% if non-KOMA class
  \IfFileExists{parskip.sty}{%
    \usepackage{parskip}
  }{% else
    \setlength{\parindent}{0pt}
    \setlength{\parskip}{6pt plus 2pt minus 1pt}}
}{% if KOMA class
  \KOMAoptions{parskip=half}}
\makeatother
\usepackage{xcolor}
\usepackage[margin=1in]{geometry}
\usepackage{color}
\usepackage{fancyvrb}
\newcommand{\VerbBar}{|}
\newcommand{\VERB}{\Verb[commandchars=\\\{\}]}
\DefineVerbatimEnvironment{Highlighting}{Verbatim}{commandchars=\\\{\}}
% Add ',fontsize=\small' for more characters per line
\usepackage{framed}
\definecolor{shadecolor}{RGB}{248,248,248}
\newenvironment{Shaded}{\begin{snugshade}}{\end{snugshade}}
\newcommand{\AlertTok}[1]{\textcolor[rgb]{0.94,0.16,0.16}{#1}}
\newcommand{\AnnotationTok}[1]{\textcolor[rgb]{0.56,0.35,0.01}{\textbf{\textit{#1}}}}
\newcommand{\AttributeTok}[1]{\textcolor[rgb]{0.77,0.63,0.00}{#1}}
\newcommand{\BaseNTok}[1]{\textcolor[rgb]{0.00,0.00,0.81}{#1}}
\newcommand{\BuiltInTok}[1]{#1}
\newcommand{\CharTok}[1]{\textcolor[rgb]{0.31,0.60,0.02}{#1}}
\newcommand{\CommentTok}[1]{\textcolor[rgb]{0.56,0.35,0.01}{\textit{#1}}}
\newcommand{\CommentVarTok}[1]{\textcolor[rgb]{0.56,0.35,0.01}{\textbf{\textit{#1}}}}
\newcommand{\ConstantTok}[1]{\textcolor[rgb]{0.00,0.00,0.00}{#1}}
\newcommand{\ControlFlowTok}[1]{\textcolor[rgb]{0.13,0.29,0.53}{\textbf{#1}}}
\newcommand{\DataTypeTok}[1]{\textcolor[rgb]{0.13,0.29,0.53}{#1}}
\newcommand{\DecValTok}[1]{\textcolor[rgb]{0.00,0.00,0.81}{#1}}
\newcommand{\DocumentationTok}[1]{\textcolor[rgb]{0.56,0.35,0.01}{\textbf{\textit{#1}}}}
\newcommand{\ErrorTok}[1]{\textcolor[rgb]{0.64,0.00,0.00}{\textbf{#1}}}
\newcommand{\ExtensionTok}[1]{#1}
\newcommand{\FloatTok}[1]{\textcolor[rgb]{0.00,0.00,0.81}{#1}}
\newcommand{\FunctionTok}[1]{\textcolor[rgb]{0.00,0.00,0.00}{#1}}
\newcommand{\ImportTok}[1]{#1}
\newcommand{\InformationTok}[1]{\textcolor[rgb]{0.56,0.35,0.01}{\textbf{\textit{#1}}}}
\newcommand{\KeywordTok}[1]{\textcolor[rgb]{0.13,0.29,0.53}{\textbf{#1}}}
\newcommand{\NormalTok}[1]{#1}
\newcommand{\OperatorTok}[1]{\textcolor[rgb]{0.81,0.36,0.00}{\textbf{#1}}}
\newcommand{\OtherTok}[1]{\textcolor[rgb]{0.56,0.35,0.01}{#1}}
\newcommand{\PreprocessorTok}[1]{\textcolor[rgb]{0.56,0.35,0.01}{\textit{#1}}}
\newcommand{\RegionMarkerTok}[1]{#1}
\newcommand{\SpecialCharTok}[1]{\textcolor[rgb]{0.00,0.00,0.00}{#1}}
\newcommand{\SpecialStringTok}[1]{\textcolor[rgb]{0.31,0.60,0.02}{#1}}
\newcommand{\StringTok}[1]{\textcolor[rgb]{0.31,0.60,0.02}{#1}}
\newcommand{\VariableTok}[1]{\textcolor[rgb]{0.00,0.00,0.00}{#1}}
\newcommand{\VerbatimStringTok}[1]{\textcolor[rgb]{0.31,0.60,0.02}{#1}}
\newcommand{\WarningTok}[1]{\textcolor[rgb]{0.56,0.35,0.01}{\textbf{\textit{#1}}}}
\usepackage{graphicx}
\makeatletter
\def\maxwidth{\ifdim\Gin@nat@width>\linewidth\linewidth\else\Gin@nat@width\fi}
\def\maxheight{\ifdim\Gin@nat@height>\textheight\textheight\else\Gin@nat@height\fi}
\makeatother
% Scale images if necessary, so that they will not overflow the page
% margins by default, and it is still possible to overwrite the defaults
% using explicit options in \includegraphics[width, height, ...]{}
\setkeys{Gin}{width=\maxwidth,height=\maxheight,keepaspectratio}
% Set default figure placement to htbp
\makeatletter
\def\fps@figure{htbp}
\makeatother
\setlength{\emergencystretch}{3em} % prevent overfull lines
\providecommand{\tightlist}{%
  \setlength{\itemsep}{0pt}\setlength{\parskip}{0pt}}
\setcounter{secnumdepth}{-\maxdimen} % remove section numbering
\usepackage{booktabs}
\usepackage{longtable}
\usepackage{array}
\usepackage{multirow}
\usepackage{wrapfig}
\usepackage{float}
\usepackage{colortbl}
\usepackage{pdflscape}
\usepackage{tabu}
\usepackage{threeparttable}
\usepackage{threeparttablex}
\usepackage[normalem]{ulem}
\usepackage{makecell}
\usepackage{xcolor}
\ifLuaTeX
  \usepackage{selnolig}  % disable illegal ligatures
\fi
\IfFileExists{bookmark.sty}{\usepackage{bookmark}}{\usepackage{hyperref}}
\IfFileExists{xurl.sty}{\usepackage{xurl}}{} % add URL line breaks if available
\urlstyle{same} % disable monospaced font for URLs
\hypersetup{
  pdftitle={How to save (and load) datasets in R: an overview},
  pdfauthor={Sascha Wolfer},
  hidelinks,
  pdfcreator={LaTeX via pandoc}}

\title{How to save (and load) datasets in R: an overview}
\author{Sascha Wolfer}
\date{2023-04-08}

\begin{document}
\maketitle

\hypertarget{primary-reference}{%
\subsubsection{Primary reference}\label{primary-reference}}

\url{http://rcrastinate.rbind.io/post/how-to-save-and-load-data-in-r-an-overview/}

\hypertarget{see-also}{%
\subsubsection{See also}\label{see-also}}

\url{https://statisticsglobe.com/r-save-load-rdata-workspace-file}
\url{https://youtu.be/svgEpRzhG7M}

\hypertarget{what-i-will-show-you}{%
\subsection{What I will show you}\label{what-i-will-show-you}}

In this post, I want to show you a few ways how you can save your
datasets in R. Maybe, this seems like a dumb question to you. But after
giving quite a few R courses mainly - but not only - for R beginners, I
came to acknowledge that the answer to this question is not obvious and
the different possibilites can be confusing. In this post, I want to
give an overview over the different alternatives and also state my
opinion which way is the best in which situation.

\hypertarget{what-are-we-going-to-do}{%
\subsection{What are we going to do?}\label{what-are-we-going-to-do}}

I will show you the following ways of saving or exporting your data from
R: - Saving it as an R object with the functions save() and saveRDS() -
Saving it as a CSV file with write.table() or fwrite() - Exporting it to
an Excel file with WriteXLS()

For me, these options cover at least 90\% of the stuff I have to do at
work. So I hope that it'll work for you, too.

\hypertarget{preparation-load-some-data}{%
\subsection{Preparation: Load some
data}\label{preparation-load-some-data}}

I will use some fairly (but not very) large dataset from the car
package. The dataset is called MplsStops and holds information about
stops made by the Minneapolis Police Department in 2017. Of course, you
can access this dataset by installing and loading the car package and
typing MplsStops. However, I want to simulate a more typical workflow
here. Namely, loading a dataset from your disk (I will load it over the
WWW). The dataset is also available from GitHub:

\begin{Shaded}
\begin{Highlighting}[]
\NormalTok{data }\OtherTok{\textless{}{-}} \FunctionTok{read.table}\NormalTok{(}\StringTok{"https://vincentarelbundock.github.io/Rdatasets/csv/carData/MplsStops.csv"}\NormalTok{,}
                   \AttributeTok{sep =} \StringTok{","}\NormalTok{, }\AttributeTok{header =}\NormalTok{ T,}
                   \AttributeTok{row.names =} \DecValTok{1}\NormalTok{)}

\NormalTok{kableExtra}\SpecialCharTok{::}\FunctionTok{scroll\_box}\NormalTok{(knitr}\SpecialCharTok{::}\FunctionTok{kable}\NormalTok{(}\FunctionTok{head}\NormalTok{(data), }\AttributeTok{row.names =}\NormalTok{ F),}
           \AttributeTok{width =} \StringTok{"100\%"}\NormalTok{, }\AttributeTok{height =} \StringTok{"300px"}\NormalTok{)}
\end{Highlighting}
\end{Shaded}

\begin{verbatim}
## Warning in !is.null(rmarkdown::metadata$output) && rmarkdown::metadata$output
## %in% : 'length(x) = 2 > 1' in coercion to 'logical(1)'
\end{verbatim}

\begin{tabular}{l|l|l|l|l|l|l|l|l|l|r|r|r|l}
\hline
idNum & date & problem & MDC & citationIssued & personSearch & vehicleSearch & preRace & race & gender & lat & long & policePrecinct & neighborhood\\
\hline
17-000003 & 2017-01-01 00:00:42 & suspicious & MDC & NA & NO & NO & Unknown & Unknown & Unknown & 44.96662 & -93.24646 & 1 & Cedar Riverside\\
\hline
17-000007 & 2017-01-01 00:03:07 & suspicious & MDC & NA & NO & NO & Unknown & Unknown & Male & 44.98045 & -93.27134 & 1 & Downtown West\\
\hline
17-000073 & 2017-01-01 00:23:15 & traffic & MDC & NA & NO & NO & Unknown & White & Female & 44.94835 & -93.27538 & 5 & Whittier\\
\hline
17-000092 & 2017-01-01 00:33:48 & suspicious & MDC & NA & NO & NO & Unknown & East African & Male & 44.94836 & -93.28135 & 5 & Whittier\\
\hline
17-000098 & 2017-01-01 00:37:58 & traffic & MDC & NA & NO & NO & Unknown & White & Female & 44.97908 & -93.26208 & 1 & Downtown West\\
\hline
17-000111 & 2017-01-01 00:46:48 & traffic & MDC & NA & NO & NO & Unknown & East African & Male & 44.98054 & -93.26363 & 1 & Downtown West\\
\hline
\end{tabular}

We now have a dataset with over 50,000 rows (you can scroll through the
first 6 of them in the box above) and 14 variables in our global
environment (the `workspace'). ust for the sake of simulating a real
workflow, I will do some very light data manipulation. Here, I'm
assigning a new column
data\(gender.not.known which is TRUE whenever data\)gender is
``Unknown'' or NA.

\begin{Shaded}
\begin{Highlighting}[]
\NormalTok{data}\SpecialCharTok{$}\NormalTok{gender.not.known }\OtherTok{\textless{}{-}} \FunctionTok{is.na}\NormalTok{(data}\SpecialCharTok{$}\NormalTok{gender) }\SpecialCharTok{|}\NormalTok{ data}\SpecialCharTok{$}\NormalTok{gender }\SpecialCharTok{==} \StringTok{"Unknown"}
\end{Highlighting}
\end{Shaded}

As I wrote above: Saving the current state of your dataset in R makes
sense when all the preparations take a lot of time. If they don't, you
can just run your pre-processing code every time you are getting back to
analyzing the dataset. In the scope of this post, let's suppose that the
calculation above took veeeery long and you absolutely don't want to run
it everytime.

\hypertarget{option-1-save-as-an-r-object}{%
\subsection{Option 1: Save as an R
object}\label{option-1-save-as-an-r-object}}

Whenever I'm the only one working on a project or everybody else is also
using R, I like to save my datasets as R objects. Basically, it's just
saving a variable/object (or several of them) in a file on your disk.
There are two ways of doing this: 1. Use the function save() to create
an .Rdata file. In these files, you can store several variables. 2. Use
the function saveRDS() to create an .Rds file. You can only store one
variable in it.

\hypertarget{option-1.1-save}{%
\subsubsection{Option 1.1: save()}\label{option-1.1-save}}

\hypertarget{you-can-save-your-data-simply-by-doing-the-following}{%
\section{You can save your data simply by doing the
following:}\label{you-can-save-your-data-simply-by-doing-the-following}}

\begin{Shaded}
\begin{Highlighting}[]
\FunctionTok{save}\NormalTok{(data, }\AttributeTok{file =} \StringTok{"data.Rdata"}\NormalTok{)}
\end{Highlighting}
\end{Shaded}

By default, the parameter compress of the save() function is turned on.
That means that the resulting file will use less space on your disk.
However, if it is a really huge dataset, it could take longer to load it
later because R first has to extract the file again. So, if you want to
save space, then leave it as it is. If you want to save time, add a
parameter compress = F.\#

(First remove the file from the Global Environment to demonstrate that
you are, in fact, reloading it)

\begin{Shaded}
\begin{Highlighting}[]
\FunctionTok{rm}\NormalTok{(data)}
\end{Highlighting}
\end{Shaded}

If you want to load such an .Rdata file into your environment, simply do

\begin{Shaded}
\begin{Highlighting}[]
\FunctionTok{load}\NormalTok{(}\AttributeTok{file =} \StringTok{"data.Rdata"}\NormalTok{)}
\end{Highlighting}
\end{Shaded}

Then, the object is available in your workspace with its old name. Here,
the new variable will also have the name data. With save() You can also
save several objects in one file. Let's duplicate data to simulate this.

\begin{Shaded}
\begin{Highlighting}[]
\NormalTok{data2 }\OtherTok{\textless{}{-}}\NormalTok{ data}

\FunctionTok{save}\NormalTok{(}\AttributeTok{list =} \FunctionTok{c}\NormalTok{(}\StringTok{"data"}\NormalTok{,}\StringTok{"data2"}\NormalTok{), }\AttributeTok{file =} \StringTok{"data.Rdata"}\NormalTok{)}
\end{Highlighting}
\end{Shaded}

Now, if you do load(``data.Rdata''), you will have two more objects in
your workspace, namely data and data2.

\begin{Shaded}
\begin{Highlighting}[]
\FunctionTok{rm}\NormalTok{(data,data2)}

\FunctionTok{load}\NormalTok{(}\AttributeTok{file =} \StringTok{"data.Rdata"}\NormalTok{)}
\end{Highlighting}
\end{Shaded}

\hypertarget{option-1.2-saverds}{%
\subsubsection{Option 1.2: saveRDS()}\label{option-1.2-saverds}}

This is the second option of saving R objects. saveRDS() can only be
used to save one object in one file. The ``loading function'' for
saveRDS() is readRDS(). Let's try it out.

\begin{Shaded}
\begin{Highlighting}[]
\FunctionTok{saveRDS}\NormalTok{(data, }\AttributeTok{file =} \StringTok{"data.Rds"}\NormalTok{)}
\NormalTok{data.copy }\OtherTok{\textless{}{-}} \FunctionTok{readRDS}\NormalTok{(}\AttributeTok{file =} \StringTok{"data.Rds"}\NormalTok{)}
\end{Highlighting}
\end{Shaded}

\hypertarget{the-difference-between-save-and-saverds}{%
\subsubsection{The difference between save() and
saveRDS()}\label{the-difference-between-save-and-saverds}}

So, you might ask ``why should I use saveRDS() instead of save()''?
Actually, I like saveRDS() better - for one specific reason that you
might not have noticed in the calls above. When we use load(), we do not
assign the result of the loading process to a variable because the
original names of the objects are used. But this also means that you
have to ``remember'' the names of the previously used objects when using
load().

When we use readRDS(), we have to assign the result of the reading
process to a variable. This might mean more typing but it also has the
advantage that you can choose a new name for the variable to integrate
it in into the rest of the new script more smoothly. Also, it is more
similar to the behavior of all the other ``reading functions'' like
read.table(): for these, you also have to assign the result to a
variable. The only advantage of save() really is that you can save
several objects into one file - but in the end it might be better to
have one file for one object. This might be more clearly organized.

(The rest of the blog entry is about saving csv and excel files. I tend
to use write.csv for the former. I do not use the WriteXLS package as
often because I prefer to output to csv, but I have used WriteLXS and it
is good to be aware of.)

\end{document}
